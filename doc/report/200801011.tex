%% Based on <bare_jrnl.tex> in the ieee package available from CTAN,
%% I have changed the options so that most useful ones are clubbed together,
%% Have a look at <bare_jrnl.tex> to understand the function of each package. 

%% This code is offered as-is - no warranty - user assumes all risk.
%% Free to use, distribute and modify.

% *** Authors should verify (and, if needed, correct) their LaTeX system  ***
% *** with the testflow diagnostic prior to trusting their LaTeX platform ***
% *** with production work. IEEE's font choices can trigger bugs that do  ***
% *** not appear when using other class files.                            ***
% Testflow can be obtained at:
% http://www.ctan.org/tex-archive/macros/latex/contrib/supported/IEEEtran/testflow

%        File: 200801011.tex
%     Created: Mon Apr 16 03:00 PM 2012 I
% Last Change: Mon Apr 16 03:00 PM 2012 I
%
\documentclass[conference]{IEEEtran}

\usepackage{cite, graphicx, subfigure, amsmath, algpseudocode, float} 
\renewcommand{\algorithmicrequire}{\textbf{Input:}}
\renewcommand{\algorithmicensure}{\textbf{Output:}}
\interdisplaylinepenalty=2500

% *** Do not adjust lengths that control margins, column widths, etc. ***
% *** Do not use packages that alter fonts (such as pslatex).         ***
% There should be no need to do such things with IEEEtran.cls V1.6 and later.


% correct bad hyphenation here
\hyphenation{}


\begin{document}
%
% paper title
\title{A Database for Semantic Data Mining}
%
%
% author names and IEEE memberships
% note positions of commas and nonbreaking spaces ( ~ ) LaTeX will not break
% a structure at a ~ so this keeps an author's name from being broken across
% two lines.
% use \thanks{} to gain access to the first footnote area
% a separate \thanks must be used for each paragraph as LaTeX2e's \thanks
% was not built to handle multiple paragraphs
\author{\IEEEauthorblockN{Nikhil Marathe}
\IEEEauthorblockA{
    Dhirubhai Ambani Institute of Information and Communication Technology\\
    Gandhinagar, Gujarat - 382007\\
    \texttt{200801011@daiict.ac.in}
}
\emph{Supervisor}\\\emph{Dr. Sourish Dasgupta}
}% <-this % stops a space
%
% The paper headers
% The only time the second header will appear is for the odd numbered pages
% after the title page when using the twoside option.


% If you want to put a publisher's ID mark on the page
% (can leave text blank if you just want to see how the
% text height on the first page will be reduced by IEEE)
%\pubid{0000--0000/00\$00.00~\copyright~2002 IEEE}

% use only for invited papers
%\specialpapernotice{(Invited Paper)}

% make the title area
\maketitle


\begin{abstract}
    We propose a system for conversion, storage and querying of semi-structured XML data from heterogenous services into structured, semantic data.
\end{abstract}

\begin{IEEEkeywords}
    Distributed databases, Query processing, Schema and sub-schema, Data mining, Knowledge management applications, Web mining, RDF, Semantic web.
\end{IEEEkeywords}

\section{Introduction}
\IEEEPARstart{T}{he} semantic web has made significant strides in the recent
past. The Linked Data initiative TODO:ref has made a huge set of data available
online, usually in very structured RDF form. Still there remains a significant
portion of data that is unstructured or semi-structured but contains valuable
information. This information is not available in a semantically aware form due
to various reasons - the data could have been produced by legacy software,
perhaps producing RDF data is resource intensive or not supported by the
creator TODO needs better reasons. This paper presents an approach to
extracting structured, semantic data from semi-structured XML \emph{from
diverse sources}, aggregating the information spread over different ontologies
and offering a SPARQL interface to access this data. A potential use case is in
scientific research, where various sensors can continuously send input to the
system and the system can present a unified interface for inferences to be
resolved from the data. Our system is designed to act in a cluster so that it
can store massive amounts of data.

\section{Background}
\label{sec:background}

TODO: diff between XML and RDF

\section{System}
Our system has the following core modules:

\begin{itemize}
    \item \emph{Clusterspace} - A graph of the various concepts spanning all
        known ontologies implemented over Neo4j TODO: ref.
    \item \emph{Triple store} - An RDF triple store implemented over Apache
        Cassandra TODO: ref.
    \item \emph{XML to RDF converter} - Attempts to generate a set of RDF
        triples from semantic information extracted from XML data.
\end{itemize}

TODO: insert diagram of system

\emph{Publishers} feed data to the system via XML or RDF. The Clusterspace is
built up as and when new ontologies are introduced to the system. Information
from the Clusterspace and the input data is used to store triples.
\emph{Subscribers} may then query the system to obtain RDF output of all known
data in the database via the SPARQL interface.

\subsection{Clusterspace}
The Clusterspace is a graph representation of a subset of the axioms stated by
all known ontologies. The Clusterspace is built up by using a \emph{reasoner}
TODO:which? the first time an ontology is introduced to the system. 

The Clusterspace stores two distinct directed trees. One is the Class heirarchy
and the other is the Property heirarchy. As new ontologies are introduced, the
reasoner is used to find the appropriate position to insert the class
(\texttt{rdf:Class}) in the Class heirarchy. The Class heirarchy has as its
roots the classes that have no super-classes. We choose not to deal with
\texttt{owl:Thing} as the single root unless an ontology explicitly introduces
it. The Property heirarchy is similar but uses the \texttt{rdf:subPropertyOf}
relationship to infer the heirarchy.

The class heirarchy also encodes object properties. An edge is created from the
domain of the property to the range(s). For datatype properties, all ranges are
co-erced to the string representation, with only
\texttt{string}\footnote{http://www.w3.org/2001/XMLSchema\#string} existing in
the Clusterspace.

The full algorithm for forming the clusterspace can be found in \ref{cs-algo}.
The advantages and limitations of building up a Clusterspace are:

\subsubsection*{Advantages}
\begin{itemize}
    \item Fast lookup of known classes using exact or approximate matches.
    \item Reasoning only has to be performed once, after that only simple graph
        traversal is required.
    \item Clusterspace can be shared and updated across a distributed system.
\end{itemize}

\subsubsection*{Limitations}
\begin{itemize}
    \item The Clusterspace does not encode full reasoning information.
    \item Complete RDFS entailment support is not currently implemented. (For
        example, relationship transitivity is not implemented yet.)
    \item Datatype information for the range is lost on Datatype Properties.
\end{itemize}

\subsection{XML to RDF conversion}
The current implementation of the XML to RDF conversion takes a very naive
approach to extracting information from the XML data. It can be immensely
augmented by using statistical properties and methods from Information
Retrieval.

The fundamental difference between plain XML and semantic data is explained in
section~\ref{sec:background}. For example consider
document~\ref{ex:xml-rec}\footnote{Based on the Music Ontology
- http://purl.org/ontology/mo/}. Here it is quite evident to a human that
Record ``OK Computer'' contains the track ``Paranoid Android'' of duration
384000ms. We use the known data properties of tags found in the data to figure
out if a datatype property reference can be resolved to a known instance
specified somewhere else in the document. We observe that the value of
a datatype property and the type of a instance can uniquely identify the
instance we are referring to\footnote{Although multiple instances can have the
    same value of a datatype property, e. g. \texttt{Person} having
    \texttt{age}, we assume that an XML representation will not use
a non-injective property to refer to instances}. Hence our lookup table is
a mapping from $(type, value)\rightarrow instance$. Although our algorithm
accepts plain XML, it has some preconditions for it to work:

\begin{itemize}
    \item The tag names should be class or property names from a known ontology.
    \item Attribute names should be a datatype property name from an known ontology.
    \item The top level tags should be class names.
    \item The conversion considers only one level of nesting in associating
        data. For example, in document~\ref{ex:xml-bad} it won't infer that the
        artist is called Radiohead.
\end{itemize}

Since these datatype property based references can be present after their use
(in terms of sequential bytes), unresolved references have to be maintained
until the entire document has been parsed, which are then resolved to create
a set of RDF triples. Based on our assumption that the top level tags are class
names, we find the right class in the clusterspace from the tag name. In case
of multiple possibilities, the implementation currently chooses the first one.
This allows us to know what properties may be defined by the inner tags. If
there are no inner tags or attributes, this XML node is itself a reference.
Otherwise this node maps to a new RDF subject instance with the
rdf:type\footnote{http://www.w3.org/1999/02/22-rdf-syntax-ns\#type} as the
class found for the tag name.

All attributes can only be datatype property instances, so we use the attribute
name to lookup the datatype property and the attribute value as the value for
this instance. A child tag can be:
\begin{enumerate}
    \item \emph{A datatype property} - In this case the child element can only
        have text content. This gives us information about the current instance
        and is added to the lookup table. A triple is also created.
    \item \emph{An object property} - If the child has only text, then the
        object in the $(subject, property, object)$ triple is a reference. This
        reference can be from any of the classes in the range of the object
        property. If the child has tags then each of those tags are treated as
        a possible description of an instance and parsed recursively. TODO:
        mention using statistics to deal with a tag defined only by its
        properties.
    \item \emph{A class} - In this case the tag is defining an instance or
        reference. Based on the known object properties, our system attempts to
        find possible object properties on the $subject$ that have the class in
        their range. The first of these is picked to create the triple.
\end{enumerate}

\begin{figure}
    \caption{Example XML document.}
    \label{ex:xml-rec}
    \begin{verbatim}
    <Track>
        <title>Paranoid Android</title>
        <duration>384000</duration>
    </Track>
    <Record title=``OK Computer''>
        <track>
            Paranoid Android
        </track>
    </Record>
    \end{verbatim}
\end{figure}

\begin{figure}
    \caption{Bad XML}
    \label{ex:xml-bad}
    \begin{verbatim}
    <MusicArtist>
        <Record>
            <title>OK Computer</title>
            <name>Radiohead</name>
        </Record>
    </MusicArtist>
    \end{verbatim}
\end{figure}

\subsection{Storing RDF in Cassandra}

TODO: What is Cassandra and Why Cassandra?

The Cassandra data model allows nested storage using super columns. We use the
approach described by Ladwig and Harth\cite{ladwig:11}, but only maintain two
models - SPO and OPS. In SPO each subject instance is a row key, the predicate
name is the supercolumn key and each object is one column. The OPS index is
used for performance improvements. We provide two additional Column-families in
our system.

\begin{enumerate}
    \item \emph{Concepts} - Each row is the full IRI of a class from the known
        ontologies. Every column is the IRI of an instance of that class. In
        addition instances of subclasses are also stored for the class.

    \item \emph{InstanceData} - Since our system is designed for data mining,
        the SPARQL response is not simply a set of triples, but full RDF dumps
        for each subject of the triples in the final result. To prevent having
        to iterate over all of an instances properties every time, a complete
        RDF/XML representation of every instance is kept in this column family,
        indexed by the IRI of the instance.

\end{enumerate}

\begin{figure}
    \caption{Clusterspace formation}
    \label{cs-algo}
    TODO this figure
\end{figure}

\begin{figure}
    \caption{Link class into Clusterspace}
    \label{cs-link-class}
    \begin{algorithmic}
        \Function{Link-Class}{$c$}
            \Require c is a previously unknown class
            \State $n\gets $ a unique node representing $c$ in the clusterspace
            \State $queue\gets $ a queue initially containing the roots of the clusterspace
            \State $linked\gets False$
            \While{not $linked$ and $queue$ is not empty}
            \State $x\gets \Call{dequeue}{queue}$
                \State $linked\gets \Call{Compare-Classes}{n,x}$
                \If{not $linked$}
                    \State Enqueue children of x
                \Else
                    exit loop
                \EndIf
            \EndWhile

            \If{not $linked$}
                \State insert $n$ as a root
            \EndIf
        \EndFunction
    \end{algorithmic}
\end{figure}

\begin{figure}
    \caption{Insert class into clusterspace at right spot}
    \label{cs-compare-classes}
    \begin{algorithmic}
        \Function{Compare-Classes}{$n$,$x$}
            \State $nClass\gets$ OWL class for Node $n$
            \State $xClass\gets$ OWL class for Node $x$
            
            \If{$nClass$ has $xClass$ as equivalent class}
                \State \Call{Add-Edge}{$n$,$x$} of type $EQUIVALENT$
                \State \Return True
            \ElsIf{$xClass$ is a superclass of $nClass$}
                \For{$subclass\gets$ each child node of $x$}
                    \If{$nClass$ has $subclass$ as equivalent class}
                        \State \Call{Add-Edge}{$n$,$subclass$} of type $EQUIVALENT$
                        \State \Return True
                    \ElsIf{$subclass$ is a subclass of $nClass$}
                        \State \Call{Remove-Edge}{$subclass$, $xClass$}
                        \State \Call{Add-Edge}{$subclass$, $nClass$}
                    \EndIf
                \EndFor

                \State \Call{Add-Edge}{$nClass$, $xClass$}

                \For{$sibling\gets$ each sibling of $x$}
                    \If{$siblingClass$ is a superclass of $nClass$ in the ontology}
                        \State \Call{Compare-Classes}{$n$, $sibling$}
                    \EndIf
                \EndFor
            \ElsIf{$xClass$ is a subclass of $nClass$}
                \State \Call{Add-Edge}{$x$, $n$}

                \For{$sibling\gets$ each sibling of $x$}
                    \If{$siblingClass$ is a superclass of $nClass$ in the ontology}
                        \State \Call{Add-Edge}{$sibling$, $n$}
                        \If{\Call{Is-Root}{$sibling$}}
                            \State \Call{Remove-Root}{$sibling$}
                            \State \Call{Add-Root}{$n$}
                        \EndIf
                    \EndIf
                \EndFor

                \If{\Call{Is-Root}{$x$}}
                    \State \Call{Remove-Root}{$x$}
                    \State \Call{Add-Root}{$n$}
                \EndIf

                \State \Return False
            \EndIf

            \State \Return False
        \EndFunction
    \end{algorithmic}
\end{figure}

\begin{figure*}
    \caption{XML to RDF conversion}
    \label{xml-to-rdf-algo}
    \begin{algorithmic}
        \Function{Convert}{$xml$}
            \Require $xml$ is a well formatted XML document
            \Ensure A set of valid RDF triples is emitted

            \State $unresolved\gets empty list$
            \State $typePropInstances\gets$ map from $(String, String) \rightarrow resource$

            \For{$element \gets$ children of root of $xml$}
                \State \Call{ToRDF}{$element$}
            \EndFor

            \For{$triple$ in $unresolved$}
                \State $resolved \gets$ \Call{Resolve}{$triple$}
                \State \Call{Emit}{$resolved$} if valid
            \EndFor
        \EndFunction

        \Function{Resolve}{$triple$}
            \Require $triple$ is $(subject, predicate, object)$
            \If{$object$ is a resource}
                \State \Return $(subject, predicate, object)$
            \ElsIf{$object$ is a Lookup}
                \State $type\gets object.type$
                \State $val\gets object.value$
                \If{\Call{Contains}{$typePropInstances$,$(type, val)$}}
                    \State \Return $(subject, predicate, typePropInstances[type,val])$
                \EndIf
                \State \Return $()$
            \EndIf
        \EndFunction
    \end{algorithmic}
\end{figure*}

\begin{figure*}
    \caption{Single element to RDF}
    \label{to-rdf-algo}
    \begin{algorithmic}
        \Function{ToRDF}{$element$}
            \State $node\gets$ Clusterspace node having class name $element$
            \If{$element$ has no attributes and no child elements}
                \State \Return \Call{Lookup}{$node$, $elementText$}
            \EndIf

            \State $subject\gets$ empty resource
            \State \Call{Emit}{$subject$, ``type'', $node$}

            \ForAll{attributes of $element$}
                \If{$attribute$ is a datatype property of $node$}
                    \State \Call{Emit}{$subject$, $property$, $attributeValue$}
                    \State \Call{Add}{$typePropInstances$, $(node,attributeValue)\rightarrow subject$}
                \EndIf
            \EndFor

            \ForAll{children of $element$}
                \If{$child$ has no children}
                    \If{$child$ is a datatype property}
                        \State \Call{Emit}{$subject$, $property$, $childText$}
                        \State \Call{Add}{$typePropInstances$, $(node,childText)\rightarrow subject$}
                    \ElsIf{$child$ is an object property}
                        \State $range\gets$ range of object property
                        \State Add $(subject,property,\Call{Lookup}{range,childText})$ to $unresolved$
                    \EndIf
                    \State Continue
                \EndIf

                \If{$child$ is an object property}
                    \ForAll{sub-children of $child$}
                        \State $val\gets$ \Call{ToRDF}{$subchild$}
                        \If{$val$ can be resolved to resource}
                            \State $valResource\gets$ \Call{Resolve}{$val$}
                            \State \Call{Emit}{$subject$, $property$, $valResource$}
                        \Else
                            \State Add $(subject,property,val)$ to $unresolved$
                        \EndIf
                    \EndFor
                    \State Continue
                \EndIf

                \If{$child$ is a class}
                    \State $property\gets$ a possible property $n\rightarrow child$
                    \State $object\gets$ \Call{ToRDF}{$child$}
                    \State Add $(subject,property,object)$ to $unresolved$
                \EndIf
            \EndFor

            \State \Return $subject$
        \EndFunction
    \end{algorithmic}
\end{figure*}

% You must have at least 2 lines in the paragraph with the drop letter
% (should never be an issue)

% needed in second column of first page if using \pubid
%\pubidadjcol

% trigger a \newpage just before the given reference
% number - used to balance the columns on the last page
% adjust value as needed - may need to be readjusted if
% the document is modified later
%\IEEEtriggeratref{8}
% The "triggered" command can be changed if desired:
%\IEEEtriggercmd{\enlargethispage{-5in}}

% references section

\bibliographystyle{IEEEtran}
\bibliography{IEEEabrv,200801011}

% biography section
% 
\end{document}

